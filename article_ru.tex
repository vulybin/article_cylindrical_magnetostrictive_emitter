\documentclass[a4paper, 12pt]{article}

\usepackage[utf8]{inputenc}
\usepackage[english, russian]{babel}
\usepackage{amssymb, amsmath}

\begin{document}

\title{К ЗАДАЧЕ ОПТИМИЗАЦИИ ПАРАМЕТРОВ МАТЕМАТИЧЕСКОЙ МОДЕЛИ ИМПУЛЬСНОГО РЕЖИМА РАБОТЫ ЦИЛИНДРИЧЕСКОГО МАГНИТОСТРИКЦИОННОГО ИЗЛУЧАТЕЛЯ}
\author{А.С. Семёнов, В.В. Улыбин}
\date{}
\maketitle

Основываясь на результатах работ [1], [2] можно поставить следующую задачу параметрической оптимизации импульсного режима работы цилиндрического магнитострикционного излучателя:

Требуется найти такие значения параметров $r_1$ и $N$, удовлетворяющие условиям  $0 < r_1 < r_2, N \geqslant 1$, (1) и такие функции $x(t), B(t)$, удовлетворяющие на отрезке $[0, T]$ системе дифференциальных уравнений
\[
\begin{cases}
x'' + \alpha_0(r_1)x' + \omega^2_0(r_1)x = f_0(r_1, B)\\
B'' + \alpha_1(r_1, \omega_1, B)B' + \omega^2_1(r_1, N, B)B = f_1(r_1, B, x),
\end{cases}
\]
начальным условиям
\[
x(0)=0, x'(0)=0, B(0)=0, B'(0)=\phi(r_1, N)
\]
и условиям
\[
\forall t \in (0, T): B(t)>0; B(T)=0,
\]
которые сообщают максимум функционалу
\[
J=\int_{0}^{T}(x'(\tau))^2d\tau
\]
Здесь $x(t)$ - изменение среднего радиуса излучателя; $B(t)$ - изменение индукции; $\alpha_0,\omega_0,f_0,\alpha_1,\omega_1,f_1$ - обозначение функций, аналитический вид которых известен.

Заметим, что функционал (5) с точностью до постоянного множителя определяет акустическую энергию излучателя в импульсе.

Сформулированная задача может быть решена только с помощью ЭВМ. Программа аналитического конструирования излучателя должна быть обеспечена достаточно полным набором тестирующих и отладочных вариантов. В настоящей работе и разрабатывается один из таких вариантов, в котором требуется найти такое значение параметра $\omega_1$, удовлетворяющее условию
\[
0 < \omega_1 < \omega_0 = const,
\]
при котором решение  системы дифференциальных уравнений
\[
\begin{cases}
x'' + 2\alpha\omega_1x' + \omega^2_0x = y\\
y'' + 2\beta\omega_1y' + \omega^2_0y = 0,
\end{cases}
\]
($\alpha$ и $\beta$ - заданные постоянные, принимающие значения из (0, 1)), удовлетворяющие начальным условиям
\[
x(0)=0, x'(0)=0, y(0)=0, y'(0)=H\beta_2,
\]
(H - заданная постоянная положительная величина), доставляет максимум функционалу
\[
J=\int_{0}^{\pi/\beta_2}(x'(\tau))^2d\tau, \beta_2=\omega_1\sqrt{1-\beta^2}.
\]

Заметим, что эта задача получается как частный случай задачи (1)-(5), если положить $y(t)=B(t), \alpha_0(r_1)=2\alpha\omega_0, \omega_0(r_1)=\omega_0, f_0(r_1, B)=y(t), \alpha_1(r_1, B, \omega_1)=2\beta\omega_1, f_1(r_1, B, x)=0, T=\pi/\beta_2$ и $\omega_1(r_1, N, B) = \omega_1$.

Решая задачу (6)-(8) при фиксированном $\omega_1$, получаем аналитическую функцию $x'(t, z) (z=\omega_1/\omega_0)$ следующего вида
\[
x'=\frac{H}{\omega_0}\frac{1}{\sqrt{\phi_1(z)}}(e^{-\beta_1t}\sin(\beta_2t+\epsilon_1)+\gamma e^{-\alpha_1t}\sin(\alpha_2t-\epsilon_2)),
\]
где 

$\alpha_1=\alpha\omega_0, \alpha_2=\omega_0\alpha_3, \beta_1=\beta\omega_1, \beta_2=\omega_1\beta_3, \gamma=\beta_3/\alpha_3,$

$\alpha_3=\sqrt{1-\alpha^2}, \beta_3=\sqrt{1-\beta^2},$

$\phi_1(z)=z^4-4\alpha\beta z^3 + (4\alpha^2 + 4\beta^2 - 2)z^2 - 4\alpha\beta z + 1;$

$\epsilon_1=\arctan\frac{(1-z^2)\beta_3}{v_1}+m_1\pi, v_1=2\alpha z-\beta(1+z^2),$

если $v_1=0$, то $\epsilon_1=\pi/2$; если $v_1>0$, то $m_1=0$; если $v_1<0$, то $m_1=1$;

$\epsilon_1=\arctan\frac{(1-z^2)\alpha_3}{v_2}+m_2\pi, v_2=2\beta z-\alpha(1+z^2),$

если $v_2=0$, то $\epsilon_2=\pi/2$; если $v_2>0$, то $m_2=0$; если $v_2<0$, то $m_2=1$;

Подставляя функцию (10) в (9) и интегрируя, получаем аналитическое задание функционала $J(\omega_1)$ как функции переменной $z (0<z<1)$ в следующем виде
\[
J=\frac{H^2}{4\omega^3_0}f(\alpha,\beta,z),
\]
где 

$
f(\alpha,\beta,z)=\frac{z}{\phi_1(z)}(\frac{1}{\beta z}(1 - \exp(-\frac{2\beta\pi}{\beta_3}))+\frac{\gamma}{\alpha}(1-\exp(-\frac{2\alpha\pi}{\beta_3 z}))-\frac{1}{z}(\sin(\epsilon_3-2\epsilon_1)+\exp(-\frac{2\beta\pi}{\beta_3})\sin(2\epsilon_1-\epsilon_3))+\frac{4\gamma}{\sqrt{(\alpha+\beta z)^2+(\alpha_3-\beta_3 z)^2}}(\sin(\epsilon_1+\epsilon_2+\epsilon_4)+\exp(-\frac{\pi(\alpha+\beta z)}{\beta_3z})\sin(\epsilon_1+\epsilon_2+\epsilon_4-\pi/(\gamma z)))-\frac{4\gamma}{\sqrt{(\alpha+\beta z)^2+(\alpha_3+\beta_3 z)^2}}(\sin(\epsilon_5+\epsilon_2-\epsilon_1)+\exp(-\frac{\pi(\alpha_\beta z)}{\beta_3z})\sin(\epsilon_5+\epsilon_2-\epsilon_1-\pi/(\gamma z)))-\gamma^2(\sin(\epsilon_6+2\epsilon_2)+\exp(-\frac{2\alpha\pi}{\beta_3z})\sin(\frac{2\pi}{\gamma z}-2\epsilon_2-\epsilon_6)))
$,

где 

$\epsilon_3=\arctan(\beta/\beta_3); \epsilon_4=\pi/2$, если $v_3=\alpha_3-\beta_3z=0,$

$\epsilon_4=\arctan\frac{\alpha+\beta z}{v_3}+m_3\pi$, $m_3=0$, если $v_3>0$, $m_3=1,$ если $v_3<0$;

$\epsilon_5=\arctan\frac{\alpha+\beta z}{\alpha_3+\beta_3 z}; \epsilon6=\arctan(\alpha/\alpha_3)$.

Следовательно, рассматриваемая задача при фиксированных значениях $\alpha$ и $\beta$ сводиться к поиску в интервале (0, 1) точки $z^*$ глобального максимума функции (11). Как показали расчеты на ЭВМ, при всех значениях $\alpha$ и $\beta$ функция (11) оказалась унимодальной. В таблице 1 в качестве примера представлены значения  функции (11) при $\alpha=0.5$, $\beta=0.001$.

\begin{table}[h]
\caption{Значения функции f(0.5, 0.001, z))}
%\begin{center}
\begin{tabular}{| c | c | c | c | c | c | c | c | c | c | c | c |}
	\hline
	$z$ & 0.10 & 0.18 & 0.26 & 0.34 & 0.42 & 0.50 & 0.58 & 0.66 & 0.74 & 0.82 & 0.90 \\
	\hline
	$f$ & 0.59 & 1.03 & 1.42 & 1.75 & 1.92 & 1.9 & 1.78 & 1.66 & 1.55 & 1.48 & 1.42 \\
	\hline
\end{tabular}
%\end{center}
\end{table}

Для определения с заданной точностью значения $z^*$ был использован алгоритм, реализующий матод золотого сечения. В таблице 2 представлены значения $z^*$ как функции параметра $\alpha$ (при $\beta=0.001$)

\begin{table}[h]
\caption{Значения функции $z^*(\alpha)$ (при $\beta=0.001$)}
%\begin{center}
\begin{tabular}{| c | c | c | c | c | c | c | c |}
	\hline
	$\alpha$ & 0.01 & 0.05 & 0.10 & 0.15 & 0.20 & 0.30 & 0.40 \\
	\hline
	$z^*$ & 0.472 & 0.474 & 0.475 & 0.475 & 0.474 & 0.469 & 0.461 \\
	\hline
	$\alpha$ & 0.50 & 0.60 & 0.70 & 0.80 & 0.90 & 0.95 & 0.99 \\
	\hline
	$z^*$ & 0.448 & 0.431 & 0.410 & 0.385 & 0.360 & 0.347 & 0.337 \\
	\hline
\end{tabular}
%\end{center}
\end{table}
\end{document}